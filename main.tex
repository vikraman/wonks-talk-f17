\environment slides

\starttext

\setvariables
  [metadata]
  [
    title={
      Homotopy theoretic\\
      aspects of\\
      \overstrike{Constructive Type Theory}\\
      Reversible Computing
    },
    author={Vikraman Choudhury},
    date={September 1, 2017},
    location={PL Wonks},
  ]

\define\PI{
  Π
}

\define[1]\loop{
  % {\Omega}\;{{#1}}
  \text{\tt Ω}\;{{#1}}
}

\define[1]\BAut{
  \text{\tt BAut}\;{{#1}}
}

\define[1]\Aut{
  \text{\tt Aut}\;{{#1}}
}

\def\Bool{
  {\blackboard{2}}
}

\def\S1{
  {\blackboard{S}}^1
}

\def\Int{
  {\blackboard{Z}}
}

\def\equiv{
  \simeq
}

\def\U{
  \mathcal{U}
}

\define[1]\sU{
  \displaystyle\sum_{X : \U}{\trunc{X = {{#1}}}}
}

\define[1]\trunc{
  \parallel{{#1}}\parallel
}

\def\Nat{
  {\blackboard{N}}
}

\define[1]\Fin{
  {\text{\tt{Fin}}}\;{{#1}}
}

\define[3]\Id{
  {{#1}}=$_{\text{\tt{{#3}}}}${{#2}}
}

\define[3]\IdU{
  {{#1}}={{#2}}
}

\define[1]\code{
  {\text{\tt{{#1}}}}
}

\define[1]\codeop{
  {\text{\tt{{#1}}}}\;
}

\def\eqdef{
  \colonequals
}

\def\pr1{
  \code{pr}_1
}

\startslide[title={Acknowledgments}]

  \startalignment[middle]
    Jacques Carette, Kyle Carter, Chao-Hong Chen, Robert Rose, Siva
    Somayyajula, Amr Sabry
  \stopalignment

\stopslide

\startslide[title={Problem}]

  \startalignment[middle]
    What is a sound and {\em complete} model for \PI?

    What does \PI have to do with Univalence?
  \stopalignment

\stopslide

\startslide[title={A likely solution}]

  \startplacetable[location=force]
    \starttabulate[|mc|Mc|mc|mc|][distance=blank]
      \HL
      \NC {\bf \PI} \NC {\bf HoTT} \NC {\bf {\infty}\:grpd} \NC {\bf space} \NC\NR
      \HL
      \TB[line]
      \NC \PI_\Bool
      \NC \sU{\Bool}
      \NC \Delta_2
      \NC K(S_2,1) \equiv {\blackboard{R}}P^{\infty}
      \NC\NR
      \TB[line]
      \NC \PI_n
      \NC \sum_{n : \Nat}{\sU{\Fin{n}}}
      \NC \coprod\limits_{n\in\Nat}\Delta_n
      \NC \oplus_{n\in\Nat}K(S_n,1) ?
      \NC\NR
      \TB[line]
      \NC \PI_{\vfrac{1}{2}}
      \NC \sU{\S1} ?
      \NC {\bf{B}}\Delta_2
      \NC ??
      \NC\NR
      \HL
    \stoptabulate
  \stopplacetable

\stopslide

\startslide[title={Classifying Spaces}]

\stopslide

\startslide[title={Notation}]

\stopslide

\startslide[title={Univalent Fibrations}]

  Let $P : A \rightarrow \U$ be a type family (fibration).

  We can \code{transport} along $P$,

  \startformula
    \codeop{transport} P : x =_A y \rightarrow P(x) \rightarrow P(y)
  \stopformula

  We can \code{transport} along $P$, but use the inverse path,

  \startformula
    \codeop{transport} P : x =_A y \rightarrow P(y) \rightarrow P(x)
  \stopformula

  This lifts a path in the base space to an equivalence in the fiber.

  \startformula
    \codeop{transport-eqv} P : x =_A y \rightarrow P(x) \equiv P(y)
  \stopformula

  $P$ is {\em univalent} if \code{transport-eqv} is an equivalence.

  \startformula
    \codeop{is-univ-fib} P \eqdef \codeop{is-equiv} (\codeop{transport-eqv} P)
  \stopformula

\stopslide

\startslide[title={Univalent Universes}]

  Let
  $\tilde{\code{U}} \eqdef \code{(}\code{U} : \U, \code{El} : \code{U}
  \rightarrow \U \code{)}$ be a Tarski universe.

  We say that $\tilde{\code{U}}$ is univalent if \code{El} is a
  univalent fibration.

\stopslide

\startslide[title={Univalent Sub-universes}]

  For any $T : \U$, the subuniverse $(\sU{T}, \pr1)$ is univalent.

  Theorem:

  \startformula
    \sU{T} \equiv \loop{\BAut{T}} \equiv \Aut{T}
  \stopformula

\stopslide

\startslide[title={Semantics of $\PI_\Bool$}]

  \startformula
    \sU{\Bool} \equiv \loop{\BAut{\Bool}} \equiv \Aut{\Bool} \equiv \Bool
  \stopformula

\stopslide

\startslide[title={Conclusion}]

  See our paper:
  \goto{https://arxiv.org/abs/1708.02710}[url(https://arxiv.org/abs/1708.02710)]

\stopslide

\stoptext

%%% Local Variables:
%%% mode: context
%%% TeX-master: t
%%% End:
