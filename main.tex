\environment slides

\input macros

\starttext

\setvariables
  [metadata]
  [
    title={
      Homotopy theoretic\\
      aspects of\\
      \overstrike{Constructive Type Theory}\\
      Reversible Computing
    },
    subtitle={
      HoTT \PI
    },
    author={
      Vikraman Choudhury
    },
    date={September 1, 2017},
    location={PL Wonks},
  ]

\startslide[title={Acknowledgments}]

  \startalignment[middle]
    Jacques Carette, Kyle Carter, Chao-Hong Chen,
    Robert Rose, Amr Sabry, Siva Somayyajula
  \stopalignment

\stopslide

\startslide[title={Problem}]

  \blank[4*big]

  \startalignment[middle]
    \rma

    What is a {\em sound} and {\em complete} model for \PI ?

    \blank[3*big]

    What does {\em completeness} have to do with {\em Univalence} ?
  \stopalignment

\stopslide

\startslide[title={A likely solution}]

  \startplacetable[location=force]
    \starttabulate[|mc|Mc|mc|mc|][distance=blank]
      \HL
      \NC {\bf \PI} \NC {\bf HoTT} \NC {\bf {\infty}\:grpd} \NC {\bf space} \NC\NR
      \HL
      \TB[line]
      \CM[background:dark] \PI_\Bool
      \CM[background:dark] {\sU{\Bool}}
      \CM[background:dark] \Delta_2
      \CM[background:dark] K(S_2,1) \equiv {\blackboard{R}}P^{\infty}
      \NC\NR
      \TB[line]
      \NC \PI_n
      \NC \sum_{n : \Nat}{\sU{\Fin{n}}}
      \NC \coprod\limits_{n\in\Nat}\Delta_n
      \NC \oplus_{n\in\Nat}K(S_n,1)
      \NC\NR
      \TB[line]
      \NC \PI_{\vfrac{1}{2}}
      \NC \sU{\S1}
      \NC {\bf{B}}\Delta_2
      \NC ??
      \NC\NR
      \HL
    \stoptabulate
  \stopplacetable

\stopslide

\startslide[title={Slogan I}]

  \startalignment[middle]
    \dontleavehmode
    \framed[corner=round,height=4em]{
      \itc{Internal logic is better than external logic!}
    }
  \stopalignment

\stopslide

\startslide[title={Prelude}]

  \blank[3*big]

  \startformula \startalign
    \NC \loop{T}
    \NC \eqdef * =_T *
    \NR
    \NC \Aut{T}
    \NC \eqdef T \equiv T
    \NR
    \NC \BAut{T}
    \NC \eqdef
    \displaystyle\sum_{X : \U}{\trunc{X \equiv T}}
    \NR
    \NC \UT{T}
    \NC \eqdef \sU{T}
    \NR
    \NC T_0
    \NC \eqdef (T , \inh{\codeop{refl} T}) : \UT{T}
    \NR
  \stopalign \stopformula

\stopslide

\startslide[title={Univalent Fibrations}]

  Let $P : A \rightarrow \U$ be a type family (fibration).

  Using \code{transport} along $P$,

  \startformula\startalign
    \NC f :
    \NC x =_A y \rightarrow P(x) \rightarrow P(y)
    \NR
    \NC g :
    \NC x =_A y \rightarrow P(y) \rightarrow P(x)
    \NR
    \NC \omega :
    \NC x =_A y \rightarrow P(x) \equiv P(y)
    \NR
  \stopalign\stopformula

  \blank[big]

  \startalignment[middle]
    \dontleavehmode
    \framed[corner=round,width=25em,height=4em]{
      $P$ is {\em univalent} if $\omega$ is an equivalence. \\
      \blank[small]
      {\em(Lumsdaine, Kapulkin, Voevodsky)} }
  \stopalignment

\stopslide

\startslide[title={Slogan II}]

  \startalignment[middle]
    \dontleavehmode
    \framed[corner=round,width=25em,height=6em]{
      \itc{The identity fibration is univalent!} \\
      \blank[big]
      \rightaligned{\ita{--- The Univalence Axiom (Voevodsky)}}
    }
  \stopalignment

\stopslide

\startslide[title={Univalent Universes}]

  Let
  $\tilde{\code{U}} \eqdef \code{(}\code{U} : \U, \code{El} : \code{U}
  \rightarrow \U \code{)}$ be a universe à la Tarski.

  \blank[3*big]

  \startalignment[middle]
    \dontleavehmode
    \framed[corner=round,width=20em,height=4em]{
      $\tilde{\code{U}}$ is {\em univalent} if \code{El} is a
      univalent fibration.
    }
  \stopalignment

\stopslide

\startslide[title={Univalent sub-universes}]

  {\em Lemma:}

  \startalignment[middle]

    For any $T : \U$, the sub-universe $(\UT{T}, \pr1)$ is univalent.

  \stopalignment

  {\em Corollary:}

  \startformula
    \loop{\UT{T}} \equiv \loop{\BAut{T}} \equiv \Aut{T}
  \stopformula

\stopslide

\startslide[title={Semantics of $\PI_\Bool$}]

  \blank[2*big]

  Using $T = \Bool$:

  \startformula
    \loop{\UT{\Bool}} \equiv \loop{\BAut{\Bool}} \equiv \Aut{\Bool}
  \stopformula

  \blank[2*big]

  Characterize $\Aut{\Bool}$:

  \startformula
    \Aut{\Bool} \equiv \Bool
  \stopformula

\stopslide

\startslide[title={Slogan III}]

  \startalignment[middle]
    \dontleavehmode
    \framed[corner=round,height=4em]{
      \itc{Equivalences are injections!}
    }
  \stopalignment

\stopslide

\startslide[title={$\Aut{\Bool} \equiv \Bool$}]

  \blank[2*big]

  If $f : \Bool \rightarrow \Bool$ is an equivalence, then $f$ is
  either \code{id} or \code{not}.

  \blank[2*big]

  \startformula \startalign[n=4,align={right,right,middle,left}]
    \NC             \NC f(\true) \NC   =  \NC f(\false) \NR
    \NC \rightarrow \NC \true    \NC   =  \NC \false    \NR
    \NC \rightarrow \NC          \NC \bot \NC           \NR
  \stopalign \stopformula

\stopslide

\startslide[title={Soundness \& Completeness}]

  \blank[4*big]

  \startalignment[middle]
    \dontleavehmode
    \startMPcode
      drawarrow (5,5)..(50,25)..(95,5) withcolor \MPcolor{foreground:dark};
      drawarrow (95,-5)..(50,-25)..(5,-5) withcolor \MPcolor{foreground:dark};
      draw (btex {\tt{Syn}} etex) shifted (-20,-3) withcolor \MPcolor{foreground:dark};
      draw (btex {\tt{Con}} etex) shifted (40,30) withcolor \MPcolor{foreground:dark};
      draw (btex {\tt{Mod}} etex) shifted (100,-3) withcolor \MPcolor{foreground:dark};
      draw (btex {\tt{Lan}} etex) shifted (40,-38) withcolor \MPcolor{foreground:dark};
      draw (btex {\ttbfc{⊥}} etex) shifted (38,-5) withcolor \MPcolor{foreground:dark};
    \stopMPcode
  \stopalignment

\stopslide

\startslide[title={Slogan IV}]

  \startalignment[middle]
    \dontleavehmode
    \framed[corner=round,height=4em]{
      \itc{Functions are Functors!}
    }
  \stopalignment

\stopslide

\startslide[title={Soundness \& Completeness}]

  {\em Level 0:}

  \startformula\startalign
    \NC ⟦\_⟧_0 :
    \NC \PI_\Bool \rightarrow \UT{\Bool}
    \NR
    \NC ⌜\_⌝_0 :
    \NC \UT{\Bool} \rightarrow \PI_\Bool
    \NR
  \stopalign\stopformula

  {\em Level 1:}

  \startformula\startalign
    \NC ⟦\_⟧_1 :
    \NC \prod_{\code{A,B}:\PI_\Bool}
    (\code{A} \leftrightarrow_1 \code{B}) \rightarrow (⟦A⟧_0 = ⟦B⟧_0)
    \NR
    \NC ⌜\_⌝_1 :
    \NC (\Bool_0 = \Bool_0) \rightarrow (⌜\Bool_0⌝_0 \leftrightarrow_1 ⌜\Bool_0⌝_0)
    \NR
  \stopalign\stopformula

  and so on \ldots

\stopslide

\startslide[title={Epilogue}]

  \useURL[arxiv][https://arxiv.org/abs/1708.02710]

  \startitemize
  \item Checkout our paper:
    \blank[medium]
    \startitemize
    \item {\ita From Reversible Programs to Univalent Universes and Back}
      \\ \from[arxiv]
    \stopitemize
  \item Follow our work on GitHub:
    \blank[medium]
    \startitemize
    \item {\tt vikraman/2DTypes}
    \item {\tt rrose1/basic-hott}
    \item {\tt ssomayyajula/HoTT}
    \item {\tt DreamLinuxer/Pi2}
    \stopitemize
  \stopitemize

\stopslide

\stoptext

%%% Local Variables:
%%% mode: context
%%% TeX-master: t
%%% End:
